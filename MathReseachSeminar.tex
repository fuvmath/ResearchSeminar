% Options for packages loaded elsewhere
\PassOptionsToPackage{unicode}{hyperref}
\PassOptionsToPackage{hyphens}{url}
%
\documentclass[
  openany]{book}
\usepackage{amsmath,amssymb}
\usepackage{iftex}
\ifPDFTeX
  \usepackage[T1]{fontenc}
  \usepackage[utf8]{inputenc}
  \usepackage{textcomp} % provide euro and other symbols
\else % if luatex or xetex
  \usepackage{unicode-math} % this also loads fontspec
  \defaultfontfeatures{Scale=MatchLowercase}
  \defaultfontfeatures[\rmfamily]{Ligatures=TeX,Scale=1}
\fi
\usepackage{lmodern}
\ifPDFTeX\else
  % xetex/luatex font selection
\fi
% Use upquote if available, for straight quotes in verbatim environments
\IfFileExists{upquote.sty}{\usepackage{upquote}}{}
\IfFileExists{microtype.sty}{% use microtype if available
  \usepackage[]{microtype}
  \UseMicrotypeSet[protrusion]{basicmath} % disable protrusion for tt fonts
}{}
\makeatletter
\@ifundefined{KOMAClassName}{% if non-KOMA class
  \IfFileExists{parskip.sty}{%
    \usepackage{parskip}
  }{% else
    \setlength{\parindent}{0pt}
    \setlength{\parskip}{6pt plus 2pt minus 1pt}}
}{% if KOMA class
  \KOMAoptions{parskip=half}}
\makeatother
\usepackage{xcolor}
\usepackage[margin=1in]{geometry}
\usepackage{longtable,booktabs,array}
\usepackage{calc} % for calculating minipage widths
% Correct order of tables after \paragraph or \subparagraph
\usepackage{etoolbox}
\makeatletter
\patchcmd\longtable{\par}{\if@noskipsec\mbox{}\fi\par}{}{}
\makeatother
% Allow footnotes in longtable head/foot
\IfFileExists{footnotehyper.sty}{\usepackage{footnotehyper}}{\usepackage{footnote}}
\makesavenoteenv{longtable}
\usepackage{graphicx}
\makeatletter
\def\maxwidth{\ifdim\Gin@nat@width>\linewidth\linewidth\else\Gin@nat@width\fi}
\def\maxheight{\ifdim\Gin@nat@height>\textheight\textheight\else\Gin@nat@height\fi}
\makeatother
% Scale images if necessary, so that they will not overflow the page
% margins by default, and it is still possible to overwrite the defaults
% using explicit options in \includegraphics[width, height, ...]{}
\setkeys{Gin}{width=\maxwidth,height=\maxheight,keepaspectratio}
% Set default figure placement to htbp
\makeatletter
\def\fps@figure{htbp}
\makeatother
\setlength{\emergencystretch}{3em} % prevent overfull lines
\providecommand{\tightlist}{%
  \setlength{\itemsep}{0pt}\setlength{\parskip}{0pt}}
\setcounter{secnumdepth}{5}
\ifLuaTeX
  \usepackage{selnolig}  % disable illegal ligatures
\fi
\usepackage[]{natbib}
\bibliographystyle{plainnat}
\IfFileExists{bookmark.sty}{\usepackage{bookmark}}{\usepackage{hyperref}}
\IfFileExists{xurl.sty}{\usepackage{xurl}}{} % add URL line breaks if available
\urlstyle{same}
\hypersetup{
  pdftitle={MATH 311: Mathematics Research Seminar - Spring 2024},
  hidelinks,
  pdfcreator={LaTeX via pandoc}}

\title{MATH 311: Mathematics Research Seminar - Spring 2024}
\author{}
\date{\vspace{-2.5em}}

\begin{document}
\maketitle

\chapter*{Syllabus}\label{syllabus}
\addcontentsline{toc}{chapter}{Syllabus}

\subsection*{Key information}\label{key-information}
\addcontentsline{toc}{subsection}{Key information}

\begin{itemize}
\tightlist
\item
  Instructors: Tan Le, Hieu Nguyen, Linh Tran, Truong-Son Van
\item
  Emails:

  \begin{itemize}
  \tightlist
  \item
    \href{mailto:tan.le@fulgright.edu.vn}{tan.le@fulbright.edu.vn}
  \item
    \href{mailto:trunghieu.nguyen@fulbright.edu.vn}{\nolinkurl{trunghieu.nguyen@fulbright.edu.vn}}
  \item
    \href{mailto:linh.tran@fulbright.edu.vn}{\nolinkurl{linh.tran@fulbright.edu.vn}}
  \item
    \href{mailto:son.van+311@fulbright.edu.vn}{son.van+310@fulbright.edu.vn}
  \end{itemize}
\item
  Class time: Fri 8AM - 11AM
\item
  Class Location: CRES CR 2
\item
  Prerequisites: at least one 300-level course
\end{itemize}

\section*{References}\label{references}
\addcontentsline{toc}{section}{References}

\begin{itemize}
\item
  Dr.~Tan Le:
\item
  Dr.~Hieu Nguyen:
\item
  Dr.~Linh Tran:
\item
  Dr.~Truong-Son Van:
\end{itemize}

\subsection*{Course description}\label{course-description}
\addcontentsline{toc}{subsection}{Course description}

This is a research seminar aimed at students who wish to major in mathematics.
In this course, mathematics faculty will first take turns presenting research problems in
their respective fields of expertise. The goal is to expose students
to different lines of research in mathematics so that interested students may
choose a potential topic to explore for Senior Capstone.
After the faculty's presentations, students will present assigned readings.
The goal of this part is to help students learn how to synthesize and convey ideas
of other peoples' work. Along the way, students will learn how to apply learned
mathematical concepts to understand complex research topics.

\subsection*{Schedule}\label{schedule}
\addcontentsline{toc}{subsection}{Schedule}

\begin{longtable}[]{@{}
  >{\raggedright\arraybackslash}p{(\columnwidth - 4\tabcolsep) * \real{0.0725}}
  >{\raggedright\arraybackslash}p{(\columnwidth - 4\tabcolsep) * \real{0.6667}}
  >{\raggedright\arraybackslash}p{(\columnwidth - 4\tabcolsep) * \real{0.2609}}@{}}
\toprule\noalign{}
\begin{minipage}[b]{\linewidth}\raggedright
Date
\end{minipage} & \begin{minipage}[b]{\linewidth}\raggedright
Topic
\end{minipage} & \begin{minipage}[b]{\linewidth}\raggedright
Presenter
\end{minipage} \\
\midrule\noalign{}
\endhead
\bottomrule\noalign{}
\endlastfoot
01/06 & Introduction to research & All faculty \\
01/06 & Introduction to partial differential equations & Dr.~Truong-Son Van \\
01/13 & Some popular topics in financial mathematics & Dr.~Tan Le \\
01/20 & & Dr.~Hieu Nguyen \\
01/27 & & Dr.~Linh Tran \\
\end{longtable}

\section*{Assessment}\label{assessment}
\addcontentsline{toc}{section}{Assessment}

During the course, students are expected to compute their own percentage
points based on the following scheme.
The instructor is not responsible for providing the running percentage.

\begin{longtable}[]{@{}cc@{}}
\toprule\noalign{}
\textbf{Form of assessment} & \textbf{Weight} \\
\midrule\noalign{}
\endhead
\bottomrule\noalign{}
\endlastfoot
Participation & 40\% \\
First presentation & 30\% \\
Second presentation & 30\% \\
\end{longtable}

The following is the letter grade breakdown. It is based on
common practice in the United States.

\begin{longtable}[]{@{}cc@{}}
\toprule\noalign{}
\textbf{Letter Grade} & \textbf{Percentage} \\
\midrule\noalign{}
\endhead
\bottomrule\noalign{}
\endlastfoot
A & {[}93,100{]} \\
A- & {[}90,93) \\
B+ & {[}87,90) \\
B & {[}83,87) \\
B- & {[}80, 83) \\
C+ & {[}77,80) \\
C & {[}73,77) \\
C- & {[}70,73) \\
D+ & {[}67,70) \\
D & {[}60, 66) \\
F & {[}0,60) \\
\end{longtable}

\subsection*{Collaboration \& Plagiarism}\label{collaboration-plagiarism}
\addcontentsline{toc}{subsection}{Collaboration \& Plagiarism}

Plagiarism is the act of submitting the intellectual property of another person as your own. It is one of the most serious of academic offenses. Acts of plagiarism include, but are not limited to:

\begin{itemize}
\item
  Copying, or allowing someone to copy, all or a part of another person's work and presenting it as your own, or not giving proper credit.
\item
  Purchasing a paper from someone (or a website) and presenting it as your own work.
\item
  Re-submitting your work from another course to fulfill a requirement in another course.
\end{itemize}

Further details can be found in the Code of Academic Integrity {[}\href{https://fulbright.edu.vn/articles/Code\%20of\%20Academic\%20Integrity/Code\%20of\%20Academic\%20Integrity_\%20Excom\%20Endorsed.pdf}{link}{]}.

\section*{Learning Support}\label{learning-support}
\addcontentsline{toc}{section}{Learning Support}

In addition to your course instructors, there are other resources available to support your
academic work at Fulbright, including one-on-one consultations with learning support staff,
supplementary workshops, and both individual and group tutoring and mentoring in course
content, language learning, and academic skills. If you would like to request learning support,
please contact Fulbright Learning Support (\url{https://learning-support.notion.site}).

\section*{Wellbeing}\label{wellbeing}
\addcontentsline{toc}{section}{Wellbeing}

Mental health and wellbeing are essential for the success of your academic journey. The
Fulbright Wellness Center provides various services including counseling, safer community,
and accessibility services. If you are experiencing undue personal or academic stress, are
feeling unsafe, or would like to know more about issues related to wellbeing, please contact
the Wellness Center via \href{mailto:wellness@fulbright.edu.vn}{\nolinkurl{wellness@fulbright.edu.vn}} or visit the Wellness Center office on
Level 5 of the Crescent campus.

For more information, pleaes check
\url{https://onestop.fulbright.edu.vn/s/article/Health-and-Wellness-Introduction}

\end{document}
